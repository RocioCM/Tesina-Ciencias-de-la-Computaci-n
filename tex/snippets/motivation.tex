\documentclass[main.tex]{subfiles}
\usepackage[utf8]{inputenc}
\usepackage[
backend=biber,
style=alphabetic,
sorting=ynt
]{biblatex}


\addbibresource{references.bib} %Imports bibliography file


\title{Motivación}
\author{Rocío Mena}
\date{\today}

\begin{document}

\maketitle

\section{Motivación}

El mundo se enfrenta a un desafío ambiental sin precedentes: la gestión insostenible de los recursos naturales. La producción y consumo masivos de bienes generan un volumen creciente de residuos, lo que pone en riesgo la salud del planeta y el bienestar de las generaciones futuras \cite{IPCC2022, pelegri2021ipcc}. En este contexto, la transición hacia una economía circular se presenta como una solución prometedora para mitigar este impacto y construir un futuro más sostenible \cite{clima2022book}. Este modelo económico busca maximizar el valor de los recursos a lo largo de su ciclo de vida, minimizando el desperdicio y reintroduciendo los materiales en los sistemas de producción \cite{da2022economia, melendez2021economia}. Sin embargo, un desafío clave para lograr una economía circular radica en la falta de transparencia y trazabilidad dentro de las cadenas de suministro tradicionales.

Esta falta de visibilidad dificulta la capacidad para identificar oportunidades de reutilización y reciclaje, responsabilizar a las industrias por su impacto ambiental y empoderar a los consumidores para que tomen decisiones informadas. Nuestra investigación tiene como objetivo abordar esta brecha mediante el desarrollo de una solución que aprovecha el poder de la tecnología blockchain para mejorar la trazabilidad de la cadena de suministro. Al mejorar la trazabilidad, podemos contribuir significativamente a la construcción de una economía circular sostenible, reducir la generación de residuos y promover prácticas de producción y consumo responsables.

\section{Trabajos existentes sobre trazabilidad y blockchain}

Investigaciones previas han explorado diversas tecnologías para mejorar la trazabilidad de la cadena de suministro, incluidos códigos de barras, etiquetas RFID y redes de sensores \cite{schuitemaker2020product}. Estas tecnologías ofrecen cierto nivel de capacidad de seguimiento; sin embargo, a menudo están limitadas por factores como la falta de estandarización, la fragmentación de información y la vulnerabilidad a la manipulación \cite{schuitemaker2020product}.

En los últimos años, la tecnología blockchain ha surgido como una solución prometedora para abordar estas limitaciones \cite{baralla2023waste, bulkowska2023implementation, alnuaimi2023blockchain}. Sus características principales, como el sistema de registro distribuido, la inmutabilidad y la transparencia, la convierten en una plataforma ideal para registrar y rastrear el movimiento de mercancías a lo largo de la cadena de suministro \cite{baralla2023waste}. Numerosos estudios han explorado diversas aplicaciones de la tecnología blockchain para la trazabilidad de la cadena de suministro, demostrando su potencial para mejorar la transparencia y la responsabilidad dentro de estos sistemas. Ejemplos de estas aplicaciones incluyen la creación de un registro inmutable del origen de los productos para verificar su autenticidad y combatir la falsificación \cite{bulkowska2023implementation}, el rastreo de materiales a lo largo de la cadena de suministro para apoyar una economía circular \cite{baralla2023waste}, la optimización de la logística y la gestión de inventario mediante información en tiempo real \cite{signeblock2024}, y la promoción de prácticas sostenibles al identificar productos con menor impacto ambiental \cite{bulkowska2023implementation}.

\section{Contribución al problema}

La investigación existente reconoce el potencial de blockchain para la trazabilidad de la cadena de suministro, pero muchas soluciones propuestas se enfocan únicamente en la tecnología blockchain \cite{baralla2023waste, bulkowska2023implementation, alnuaimi2023blockchain}. Nuestro trabajo va más allá al proponer un enfoque híbrido que integra blockchain con Internet de las cosas (IoT) y sistemas de gestión tradicionales. Esta integración nos permite aprovechar los datos en tiempo real de los sensores de IoT, proporcionando una visión más completa y confiable del movimiento y el estado del producto a lo largo de la cadena de suministro. Además, nuestra solución incorpora sistemas de gestión tradicionales, asegurando la compatibilidad y facilitando la adopción dentro de las prácticas comerciales existentes. Creemos que este enfoque combinado ofrece una implementación más factible y práctica para mejorar la trazabilidad de la cadena de suministro, en última instancia, contribuyendo a la transición hacia una economía circular sostenible.

A su vez, este trabajo se enfoca específicamente en la cadena de suministro y reciclaje de vidrio. Esta decisión se fundamenta en la importancia del vidrio como material reciclable y la necesidad de mejorar su gestión dentro de la economía circular. En Latinoamérica, el 5\% de los residuos sólidos urbanos son vidrio \cite{cepal2021economia}, y solo el 20\% de este vidrio se recicla \cite{verallia2022whitebook}. La baja tasa de reciclaje de vidrio en la región se debe a la falta de infraestructura y sistemas de gestión adecuados, así como a la falta de conciencia y educación sobre la importancia del reciclaje. Al mejorar la trazabilidad en la cadena de suministro del vidrio, podemos facilitar su reciclaje y promover una economía circular más sostenible en la región. Al visibilizar el flujo de materiales y promover prácticas de reciclaje, y facilitar la información y procesos a los usuarios, podemos reducir la generación de residuos, disminuir la extracción de materias primas vírgenes y fomentar la reutilización de materiales en la producción de nuevos envases de vidrio.

Particularmente enmarcados en la actividad económica principal de nuestra provincia, la producción de vino, esta es una problemática local y concreta que puede ser abordada con una solución tecnológica y tener un impacto real en la economía local. En Mendoza, la producción de vino es una actividad económica importante y la industria del vidrio es un actor clave en la cadena de suministro de vino al proveer los envases para el embotellado de los vinos. Por lo tanto, mejorar la trazabilidad en la cadena de suministro del vidrio puede tener un impacto significativo en la sostenibilidad de la industria vitivinícola y en la economía regional. 

Al abordar este caso de estudio específico, esperamos proporcionar una solución concreta y aplicable en el ecosistema mendocino que a su vez pueda servir en un futuro como modelo para adaptarse a otras industrias y otros materiales reciclables.

\end{document}
