\documentclass[main.tex]{subfiles}
\usepackage[utf8]{inputenc}
\usepackage[
backend=biber,
style=alphabetic,
sorting=ynt
]{biblatex}


\addbibresource{references.bib} %Imports bibliography file


\title{Introducción}
\author{Rocío Mena}
\date{\today}

\begin{document}

\maketitle

\section{Introducción}

El mundo se enfrenta a un desafío ambiental sin precedentes: la gestión insostenible de los recursos naturales. La producción y consumo masivos de bienes generan un volumen creciente de residuos, lo que pone en riesgo la salud del planeta y el bienestar de las generaciones futuras \cite{IPCC2022, pelegri2021ipcc}. En este contexto, la transición hacia una economía circular se presenta como una alternativa viable para mitigar este impacto y construir un futuro más sostenible \cite{clima2022book}.

La economía circular es un modelo de producción y consumo que busca maximizar el valor de los recursos a lo largo de su ciclo de vida, minimizando la generación de residuos y reincorporando los materiales al sistema productivo para extender su vida útil \cite{da2022economia, melendez2021economia}. Para lograr este objetivo, es fundamental contar con mecanismos eficientes de trazabilidad que permitan rastrear el flujo de materiales desde su origen hasta su disposición final.

Las cadenas de suministro constituyen el conjunto de procesos y actores involucrados en la acción de llevar un producto o servicio desde su origen hasta el consumidor final. Las cadenas de suministro tradicionales se caracterizan por una falta de transparencia y trazabilidad en el flujo de materiales. Esto dificulta la identificación de oportunidades para la reutilización y el reciclaje, y limita la capacidad de responsabilizar a las industrias por el impacto ambiental de sus productos. Las prácticas actuales de trazabilidad suelen basarse en sistemas manuales y fragmentados, que son propensos a errores y manipulaciones. Además, estos sistemas no suelen estar integrados a lo largo de toda la cadena de suministro, lo que dificulta la obtención de una visión completa del ciclo de vida de los materiales.

Específicamente en el caso del vidrio, su alta reciclabilidad y bajo costo de procesamiento \cite{prodvidrio2024verallia} presentan una oportunidad perfecta para la trazabilidad efectiva en la cadena de suministro. El vidrio es un material clave en la industria del envasado de bebidas y alimentos y a su vez puede reinsertarse en el ciclo productivo sin perder calidad o pureza. La mejora en los sistemas de trazabilidad no solo facilita la identificación y segregación del vidrio reciclable, sino que también optimiza los procesos de recogida y reciclaje. A su vez, este sistema presenta la oportunidad de generar un circuito tanto de reutilización como de reciclaje de envases de vidrio. Este enfoque disminuye la necesidad de extracción de materias primas vírgenes y disminuye el gasto energético asociado a la producción de nuevos envases, contribuyendo significativamente a la reducción de la huella de carbono y a la promoción de prácticas de producción más limpias. Este ciclo virtuoso no solo beneficia al medio ambiente, sino que también ofrece ventajas económicas a las empresas al reducir los costos asociados con la producción de nuevos envases  \cite{prodvidrio2024verallia}, lo que a su vez puede traducirse en precios más bajos para los consumidores.

La trazabilidad en la cadena de suministro juega un papel fundamental en la transición hacia una economía circular. Al permitir rastrear el flujo de materiales desde su origen hasta su disposición final, la trazabilidad puede ayudar a:

\begin{enumerate}
	\item Identificar oportunidades para la reutilización y el reciclaje: la trazabilidad permite identificar qué materiales pueden ser reutilizados o reciclados en diferentes etapas de la cadena de suministro. Esto puede ayudar a reducir la generación de residuos y aumentar la eficiencia en el uso de recursos.

	\item Responsabilizar a las industrias por su impacto ambiental: la trazabilidad permite rastrear el origen de los materiales y el impacto ambiental de su producción y consumo. Esto puede ayudar a responsabilizar a las industrias por su huella ambiental y promover prácticas más sostenibles \cite{melendez2021economia}.

	\item Promover la transparencia en las cadenas de suministro: la trazabilidad puede ayudar a aumentar la transparencia en las cadenas de suministro, lo que puede generar confianza entre los consumidores y las empresas.
\end{enumerate}

La tecnología blockchain ofrece una solución para la trazabilidad en la cadena de suministro. Una cadena de bloques o blockchain es un registro distribuido e inmutable que permite almacenar y compartir información de manera segura y transparente entre todos los participantes de la cadena \cite{rennock2018blockchain}. Las características de la tecnología blockchain, como su inmutabilidad, transparencia y trazabilidad, la convierten en una herramienta ideal para la gestión sostenible de la cadena de suministro \cite{baralla2023waste, bulkowska2023implementation, alnuaimi2023blockchain}.

En este trabajo de tesis, proponemos una solución para mejorar la trazabilidad en la cadena de suministro utilizando tecnología blockchain integrada con otras tecnologías complementarias como Internet de las Cosas (IoT) y sistemas de gestión tradicionales. La propuesta se basa en el desarrollo de una plataforma blockchain fácilmente integrable con sistemas externos para la carga y lectura de información sobre productos y materias primas a lo largo de la cadena de suministros. Esta plataforma es capaz de incorporar datos de distintos sensores IoT de forma confiable y automatizada en varias etapas de la cadena de suministros. A su vez, se puede incorporar datos a la plataforma de forma manual o automatizada a través de sistemas informáticos de gestión utilizando una API REST, facilitando su adopción. Esta plataforma proporciona una visibilidad completa en tiempo real del movimiento y el estado de los productos a lo largo de la cadena de suministro.

La plataforma blockchain permitirá crear un registro inmutable y transparente de todos los eventos que ocurren en la cadena de suministro, desde la producción hasta el consumo. Existe la posibilidad futura de extender el uso de este sistema luego del fin de la vida útil del producto, durante la etapa de disposición final y posible revalorización de las materias primas del producto.

Esperamos que la implementación de esta solución contribuya significativamente a la transición hacia una economía circular sostenible. Al mejorar la trazabilidad en la cadena de suministro, se puede reducir la generación de residuos, responsabilizar a las industrias por su impacto ambiental y promover prácticas más sostenibles en la producción y el consumo de bienes.

Esta tesis se estructura en varias secciones: la Introducción (Sección 1) expone la problemática de la trazabilidad en la cadena de suministro, la propuesta de solución y las posibles limitaciones. La Metodología (Sección 2) de investigación describe detalladamente el enfoque que se utilizará para desarrollar y evaluar la plataforma blockchain propuesta. La Revisión de la literatura (Sección 3) expone el estado actual del arte e investigación existente sobre trazabilidad en la cadena de suministro y tecnologías blockchain. El Diseño e implementación de la plataforma blockchain (Sección 4) detalla la arquitectura del sistema, los componentes principales y las funcionalidades. La Evaluación de la plataforma blockchain (Sección 5) presenta los resultados de implementación y evaluación del impacto ambiental. Finalmente, las Conclusiones y trabajo futuro (Sección 5) resumen los hallazgos del estudio y sugieren direcciones para investigaciones futuras.

\end{document}
